%==============================================================================
% PACKAGE IMPORTS
%==============================================================================

% === Document Structure and Layout ===
\usepackage[utf8]{inputenc}           % For UTF-8 encoding
\usepackage{geometry}                 % For adjusting page layout and margins
\usepackage{fancyhdr}                 % For customizing page headers and footers
\usepackage{layout}                   % Use `\layout` to visualize the layout
\usepackage{titlesec}                 % For customizing section titles
\usepackage{float}                    % Positioning floating elements
\usepackage{tocloft}                  % For customizing the Table of Contents
\usepackage{setspace}                 % For line spacing
\usepackage{framed}                   % For framed environment

% === Mathematics, Symbols and Technical Content ===
\usepackage{amsmath}                  % Enhanced math typesetting
\usepackage{amssymb}                  % Additional math symbols
\usepackage{tikz-cd}                  % For commutative diagrams

% === Graphics, Figures and Diagrams ===
\usepackage{graphicx}                 % For graphics/images
\usepackage{tikz}                     % For vector graphics
\usepackage{tikz-3dplot}              % For 3D graphics

% === Tables and Arrays ===
\usepackage{array}                    % Enhanced tabular environments

% === Code Listings and Algorithms ===
\usepackage{algorithm}                % For algorithm blocks
\usepackage{algpseudocode}            % For pseudocode
\usepackage{listings}                 % For code listings
\usepackage{verbatim}                 % For multiline comments

% === Colors and Styling ===
\usepackage{xcolor}                   % Extended color support
\usepackage{color}                    % For color definitions
\usepackage[fitting]{tcolorbox}       % For colored boxes

% === References, Hyperlinks and Cross-Referencing ===
\usepackage{xurl}                     % For URLs
% \usepackage[
%     acronyms,       % For acronyms
%     acronym,        % For acronyms
%     automake,       % Automatically make glossaries
%     toc,            % Add glossaries to the Table of Contents
%     acronyms,       % Enable acronyms
%     nonumberlist    % Suppress page numbers in the glossary
% ]{glossaries}
% \makeglossaries                       % Initialize glossaries
\usepackage{bookmark}                 % Better PDF bookmarks (fixes warnings)
\usepackage{hyperref}                 % Enables hyperlinks and PDF metadata (bookmarks option removed)

% === Text Formatting and Typography ===
\usepackage[none]{hyphenat}           % Disables hyphenation throughout the document
\usepackage{pbox}                     % For parbox with line breaks
\usepackage{lmodern}                  % For Latin Modern needed for tcolorbox
\usepackage[T1]{fontenc}              % Use T1 font encoding for better character support
% \usepackage{textcomp}               % Provides \textquotedbl command
% \usepackage{mathptmx}               % Times New Roman font for math
% \usepackage{txfonts}                % Times New Roman font for text and math
\usepackage{times}                    % Times New Roman font for text
\renewcommand{\rmdefault}{ptm}      % Set Times New Roman as the default font


% === Miscellaneous Utilities ===
\usepackage{lipsum}                   % For placeholder text
\usepackage[nodayofweek]{datetime}    % For date formatting
\usepackage{etoolbox}                 % For patching commands
\usepackage{pgffor}                   % For loop constructs
\usepackage[useregional]{datetime2}
