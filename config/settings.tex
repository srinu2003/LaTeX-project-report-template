%==============================================================================
% CONFIGURATION SETTINGS
%==============================================================================
% These settings control the appearance and behavior of various document elements.
% Modify these settings to match your institution's requirements or your preferences.

% === TikZ and Graphics Configuration ===
\usetikzlibrary{external, shapes.geometric, arrows, arrows.meta, decorations.pathmorphing, calc}

% \usetikzlibrary{
%     arrows, arrows.meta, automata, backgrounds, calendar, 
%     chains, decorations, external, fadings, fit, 
%     matrix, patterns, patterns.meta, petri, 
%     positioning, scopes, shadows, shapes, 
%     shapes.geometric, shapes.arrows, shapes.symbols, 
%     shapes.misc, shapes.multipart, shapes.callouts,
%     spy, trees, turtle,
%     decorations.pathmorphing, decorations.pathreplacing, 
%     decorations.markings, decorations.footprints, 
%     decorations.shapes, decorations.text,
%     calc, fixedpointarithmetic, fpu, intersections, 
%     math, through,
%     3d, perspective,
%     circuits, circuits.logic, circuits.ee,
%     graphs, graphs.standard, 
%     er, folding, lindenmayersystems, 
%     mindmap, plothandlers, quotes, svg.path, topaths,
%     babel
% }
\graphicspath{{images/}{./}}          % Set paths for images - both in images/ folder and root directory

% === Color Definitions ===
% These colors are used throughout the document for various elements
% Modify them to match your institution's color scheme if necessary
\definecolor{deepgray}{rgb}{0.4, 0.4, 0.4}    % Dark gray
\definecolor{deepblue}{rgb}{0,0,0.6}          % Dark blue
\definecolor{deepred}{rgb}{0.6,0,0}           % Dark red
\definecolor{deepgreen}{rgb}{0,0.5,0}         % Dark green
\definecolor{codegreen}{rgb}{0,0.6,0}         % Green for comments
\definecolor{codegray}{rgb}{0.5,0.5,0.5}      % Gray for line numbers
\definecolor{codepurple}{rgb}{0.58,0,0.82}    % Purple for strings
\definecolor{backcolour}{rgb}{0.95,0.95,0.92} % Light background
\definecolor{brown}{HTML}{612322}             % Brown for headers/footers
\definecolor{darkred}{HTML}{C00000}          % Dark red for titles
% Project Report Title Styling
\newtcolorbox{titlebox}{
    colframe=white,             % Frame color white (invisible)
    colback=white,              % Background color white (invisible)
    boxrule=0pt,                % No border
    left=0pt,                   % No left padding
    right=0pt,                  % No right padding
    top=0pt,                    % No top padding
    bottom=0pt,                 % No bottom padding
    width=\textwidth,           % Full width of the text area
    fit to height=1in,          % Height of the box
    halign=center,              % Center the text horizontally
    valign=center,              % Center the text vertically
    fonttitle=\Huge\bfseries,   % Title font size and style
    fit basedim=26pt,           % Fit the box to the text height
    fontupper=\Huge\bfseries,   % Text font size and style
    coltitle=black,             % Title color
}

% === Code Listing Styles ===
\lstdefinestyle{pythonstyle}{
    keywords={False, await, else, import, pass, None, break, except, in, raise, True, class, finally, is, return, and, continue, for, lambda, try, as, def, from, nonlocal, while, assert, del, global, not, with, async, elif, if, or, yield}, % Python keywords
    keywordstyle=\color{deepblue}\bfseries,
    ndkeywords={self},
    ndkeywordstyle=\color{deepgray}\bfseries,
    identifierstyle=\color{black},
    sensitive=true,
    comment=[l]{\#},
    commentstyle=\color{deepgreen}\ttfamily,
    stringstyle=\color{deepred}\ttfamily,
    numberstyle=\tiny\color{gray},
    morestring=[b]',
    morestring=[b]``,
    basicstyle={
        \ttfamily
        \hyphenpenalty=50
        \exhyphenpenalty=50
    },
    breakatwhitespace=false,
    breaklines=true,
    captionpos=b,
    keepspaces=true,
    numbers=left,
    numbersep=5pt,
    showspaces=false,
    showstringspaces=false,
    showtabs=false,
    tabsize=2
}

\lstdefinestyle{graystyle}{
    backgroundcolor=\color{backcolour},
    commentstyle=\color{codegreen},
    keywordstyle=\color{magenta},
    numberstyle=\tiny\color{codegray},
    stringstyle=\color{codepurple},
    basicstyle={\ttfamily\hyphenpenalty=50\exhyphenpenalty=50},
    comment=[l]{\#},
    commentstyle=\color{deepgreen}\footnotesize\rmfamily,
    breakatwhitespace=false,
    breaklines=true,
    captionpos=b,
    keepspaces=true,
    numbers=left,
    numbersep=5pt,
    showspaces=false,
    showstringspaces=false,
    showtabs=false,
    tabsize=2
}

\lstset{style=graystyle}

% Set page geometry with 1-inch margins on A4 paper
\geometry{margin=1in}  % Uncomment to set 1-inch margins

% Configure hyperref for proper cross-references and bookmarks
\hypersetup{
    breaklinks=true,            % allow line breaks in links
    pdfstartview={FitH},        % fits the width of the page to the window
    pdftitle={\docTitle},       % title
    pdfauthor={\pdfauthor},     % author
    pdfsubject={\pdfsubject},   % subject of the document
    pdfcreator={\studentlist},  % creator of the document
    pdfkeywords={\pdfkeywords}, % list of keywords
    colorlinks=false,           % false: boxed links; true: colored links
    linkcolor=deepblue,         % color of internal links
    citecolor=deepred,          % color of links to bibliography
    filecolor=deepgreen,        % color of file links
    urlcolor=blue               % color of external links
}

% Typography settings
\sloppy
\hfuzz=99pt                     % Suppress warnings for overfull hboxes
\vfuzz=99pt                     % Suppress warnings for overfull vboxes
\hbadness=10000                 % Suppress warnings for underfull hboxes
\exhyphenpenalty=10000          % Suppress hyphenation across lines
\hyphenpenalty=10000            % Suppress hyphenation across lines
% \raggedbottom                   % Allow pages to be ragged at the bottom

\renewcommand{\cftbeforetoctitleskip}{-.5em}
% \renewcommand{\cftaftertoctitleskip}{1em}
\renewcommand{\contentsname}{\hfil Contents}
\addtocontents{toc}{
    \textbf{S .No. \leftskip2cm Chapter Name \hfill Page No.}\par
}   % Add 'S. No.', 'Name of Chapter', and 'Page No.' to the ToC

% Add a horizontal line after the Table of Contents header
\addtocontents{toc}{
    \protect\mbox{}\protect\hrulefill    
}% Add a horizontal line after the ToC header

% %==============================================================================
% % ACRONYM DEFINITIONS
% %==============================================================================
% % Define acronyms that will be used throughout the document
% % Format: \newacronym{label}{abbreviation}{full form}
% \newacronym{ai}{AI}{Artificial Intelligence}
% \newacronym{ml}{ML}{Machine Learning}
% \newacronym{nlp}{NLP}{Natural Language Processing}


% %===============================================================================
% % ADDITIONAL SETTINGS
% %===============================================================================
% % Add any additional settings or configurations here
% % Chpater Tittle Formatting
% \titleformat{\chapter}[display]
%     {\normalfont\huge\bfseries\raggedleft}
%     {\vspace{-40pt}                                 % Space before chapter number
%     \chaptertitlename\ \thechapter}
%     {5pt}                                           % Reduced space between chapter number and title
%     {\Huge}                                         % Title font size
%     []                                              % No additional code after title
